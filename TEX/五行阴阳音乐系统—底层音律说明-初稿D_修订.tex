% !TEX TS-program = xelatex
\documentclass{article}

\usepackage{fontspec}
\usepackage{amsmath,amssymb}
\usepackage{xeCJK}
\usepackage[normalem]{ulem}
\usepackage{geometry}
\geometry{a4paper, margin=25mm}

\setmainfont{Kode Mono}
\setCJKmainfont{Songti SC}[ItalicFont={Songti SC}]
\setCJKmonofont{Songti SC}[ItalicFont={Songti SC}]

\newcommand{\emphzh}[1]{\uline{#1}}

\title{Petersen 五行阴阳音乐系统 — 底层音律说明}
\author{Abe.Chua (初稿D — 修订,匹配黄金率试验)}
\date{2025-08-25}

\begin{document}
\maketitle

本修正版将文档中的映射与常量修正为与“黄金率试验 — 初稿D”一致:
360° 对应一个黄金率音区,五行间隔为 72°,金(e=0)阴(p=-1)对应 0° 基位,阴阳极性在格内的偏移默认 $\Delta\theta=4.8^\circ$,并采用按 $\varphi$ 的区间归一化(方案 B)为首选实现方式。

\section{概览与设计目标(保持不变)}
系统以黄金比例为音高生成基准,并以 Petersen 图与五行/阴阳结构作为语法。目标包括参数化、可复现的音律生成(15 方位,扩展至多音区)、便于导出 Scala/.scl 与 MIDI Tuning、以及支持和弦评分与实时合成。

\section{符号与默认常量(修正)}
主要符号与默认值:
\[
\varphi=\frac{1+\sqrt5}{2}\approx1.61803398875,\qquad F_{\mathrm{base}}=20\ \mathrm{Hz}\ (\text{默认}).
\]
五行角度基位与极性:
\[
\theta_e = 72^\circ\cdot e,\quad e\in\{0,1,2,3,4\}\quad(0=\text{金}),
\]
\[
p\in\{-1,0,+1\},\qquad \Delta\theta = 4.8^\circ\ (\text{默认}).
\]
阴阳在格内的角度映射(使 p=-1 对应基位角 \(\theta_e\)):
\[
\theta = \theta_e + (p+1)\cdot\Delta\theta,\qquad p\in\{-1,0,+1\}.
\]

备注:上述定义保证金阴(e=0,p=-1)角度为 \(0^\circ\)。

\section{核心映射(与黄金率试验一致)}
将角度按 360° 周期映射到黄金率比例因子。原始比例因子:
\[
r_{\mathrm{raw}}(\theta)=\varphi^{\theta/360}.
\]
原始频率(未归一化):
\[
f_{\mathrm{raw}} = F_{\mathrm{base}}\cdot r_{\mathrm{raw}}(\theta)=F_{\mathrm{base}}\cdot\varphi^{\theta/360}.
\]

推荐优先使用“方案 B(区内归一化)”——把每 360° 映射为单一音区的相对位置:
\[
u=\frac{(\theta \bmod 360^\circ)}{360^\circ}\in[0,1).
\]
第 \(n\) 音区定义为:
\[
\text{区}_n=[\,F_{\mathrm{base}}\varphi^{n},\ F_{\mathrm{base}}\varphi^{n+1}\,).
\]
在区内按 \(u\) 放置音位:
\[
f_{e,p,n} = F_{\mathrm{base}}\cdot \varphi^{\,n+u}.
\]

方案 A(折叠)仍可保留作兼容用途:先计算 \(f_{\mathrm{raw}}=F_{\mathrm{base}}\varphi^{\theta/360}\),再乘以 \(\varphi^{k}\) 或 \(R_{\mathrm{oct}}^{k}\) 折叠到目标区间;但推荐以方案 B 作为默认实现以避免折叠二义性。

\section{每度 cents 增量(修正)}
当 360° 对应一个完整的 \(\varphi\) 倍程时,每 1° 对应的 cents 增量为:
\[
\Delta_{1^\circ} = \frac{1200}{360}\log_2\varphi
= \frac{10}{3}\log_2\varphi \approx 2.3147\ \text{cents/deg}.
\]

\section{如何枚举有效的 \(n\)}
为了使频率落在工程带宽 \([F_{\min},F_{\max}]\),对给定 \(u\):
\[
F_{\min}\le F_{\mathrm{base}}\varphi^{n+u}<F_{\max}
\]
等价于
\[
\log_{\varphi}\frac{F_{\min}}{F_{\mathrm{base}}}-u \le n < \log_{\varphi}\frac{F_{\max}}{F_{\mathrm{base}}}-u,
\]
因此可取
\[
n_{\min}=\left\lceil \log_{\varphi}\frac{F_{\min}}{F_{\mathrm{base}}}-u \right\rceil,\quad
n_{\max}=\left\lfloor \log_{\varphi}\frac{F_{\max}}{F_{\mathrm{base}}}-u \right\rfloor.
\]

对所有 15 个 \((e,p)\) 计算对应 \(u\),再合并不同 \(n\) 的结果可得到完整候选集合。

\section{15 方位、45 音区与默认参数(修正)}
15 方位仍为 五行(5)× 三极性(3)。默认参数更新以匹配黄金率试验文档:
\begin{itemize}
  \item \(F_{\mathrm{base}} = 20\ \mathrm{Hz}\)(默认,可按需要调整为 220 Hz 等)
  \item \(\Delta\theta = 4.8^\circ\)
  \item 映射周期:360° → 一完整 \(\varphi\) 区间(使用方案 B)
\end{itemize}

可通过对每个 15 方位并对若干 \(n\)(由带宽约束)生成最终频率表;若需要 45 音区,可在每个基本 \(u\) 上枚举多值 \(n\)(低/中/高区)。

\section{导出与 DAW/合成器 集成(简要提醒)}
建议输出 CSV、Scala (.scl) 与 MIDI Tuning (.tun) 三种格式。注意:
\begin{itemize}
  \item 若选择 \(F_{\mathrm{base}}=20\ \mathrm{Hz}\),导出时可同时给出相对 cents(以本次选定基准如 220Hz 或区下界为准)。
  \item MPE/pitch-bend 分配策略与先前建议一致,但 cents 计算需基于本修正后的频率值。
\end{itemize}

\section{和弦评分、搜索、实时与实现注意(保持)}
和弦评分、搜索与实时合成的实现注意点保持不变;在所有频率/cent 计算处使用本修正后的频率公式与每度 cents 增量。

\section{附录:更新的最小 Python 脚本(工程化示例,匹配修正公式)}
下面脚本与原附录脚本等效,但已修正为:360° 对应完整 \(\varphi\) 倍程,阴阳极性按 \(\theta=\theta_e+(p+1)\Delta\theta\),默认 \(F_{\mathrm{base}}=20\ \mathrm{Hz}\),\(\Delta\theta=4.8^\circ\),并采用方案 B(区内归一化枚举)。将脚本保存在 \texttt{tools/generate\_tuning.py} 并在 macOS 终端运行 \texttt{python3}。

\begin{verbatim}
# filepath: tools/generate_tuning.py
# 运行: python3 tools/generate_tuning.py
import math, csv

phi = (1 + 5**0.5) / 2
F_base = 20.0          # 默认基点频率(与黄金率试验一致)
delta_theta = 4.8      # 默认角度偏移(deg)
F_min = 30.0
F_max = 6000.0

elements = ['金','木','水','火','土']
rows = []

def theta_for(e_idx, p):
    theta_e = 72.0 * e_idx
    return theta_e + (p + 1) * delta_theta  # p=-1 -> theta_e

def u_from_theta(theta):
    return (theta % 360.0) / 360.0

def freq_in_zone(n, u):
    return F_base * (phi ** (n + u))

for e_idx, name in enumerate(elements):
    for p in (-1, 0, 1):
        theta = theta_for(e_idx, p)
        u = u_from_theta(theta)
        # 枚举满足带宽的 n
        n_min = math.ceil(math.log(F_min / F_base, phi) - u)
        n_max = math.floor(math.log(F_max / F_base, phi) - u)
        for n in range(n_min, n_max + 1):
            f = freq_in_zone(n, u)
            cents = 1200 * math.log2(f / F_base)
            rows.append({
                'element': name,
                'e': e_idx,
                'p': p,
                'theta': round(theta,6),
                'n': n,
                'u': round(u,6),
                'freq': round(f,6),
                'cents': round(cents,3)
            })

# CSV
with open('tools/15_positions_expanded.csv','w', newline='') as f:
    w = csv.DictWriter(f, fieldnames=['element','e','p','theta','n','u','freq','cents'])
    w.writeheader()
    w.writerows(rows)

# 简单 Scala .scl 以 F_base 为参考(写入相对于 F_base 的 cents)
with open('tools/petersen_15_phi.scl','w') as f:
    f.write("! petersen_15_phi.scl\n")
    f.write("Petersen 五行阴阳 15-tone scale (F_base=20Hz, delta_theta=4.8deg)\n")
    f.write(str(len(rows)) + "\n")
    for r in rows:
        f.write(f"{r['cents']}\n")

print("Generated tools/15_positions_expanded.csv and tools/petersen_15_phi.scl")
\end{verbatim}

\section{结语}
本修订消除了原底层说明中将 360° 映射为 72° 分母(以及不同默认基频/偏移)造成的不一致。现在文档在角度—频率映射、默认常量与导出脚本上与“黄金率试验 — 初稿D”保持一致。

\end{document}