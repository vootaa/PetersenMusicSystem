% !TEX TS-program = xelatex
\documentclass{article}
\usepackage{fontspec}
\usepackage{xeCJK}
\usepackage{textcomp}
\usepackage{geometry}
\usepackage{hyperref}
\usepackage{amsmath}
\geometry{a4paper, margin=25mm}

\setmainfont{Kode Mono}
\setCJKmainfont{Songti SC}

\title{基于 Petersen 图启发的五行阴阳音乐系统:理论框架与创新潜力}
\author{Abe.Chua}
\date{2025-08-24}

\begin{document}
\maketitle

\section*{研究立项报告书(初稿)}
\textbf{作者}:Abe.Chua \\
\textbf{日期}:2025年8月24日 \\
\textbf{关键词}:Petersen Graph、五行阴阳、黄金比例调律、非八度音阶、实验音乐系统

\section{摘要}
本报告提出一种基于 Petersen 图拓扑结构的原创音乐系统,融入中国传统五行(金、木、水、火、土)与阴阳极性哲学,构建脱离 12-TET 的非线性音律框架。核心包括 15 方位音阶(5 五行 × 3 极性),可扩展至 45 音符(3 音区 × 15 方位),通过黄金比例缩放与图约束生成旋律。系统强调规则驱动的和谐探索,潜在产生新颖且悦耳的音乐作品。报告讨论理论基础、创新点、可行性分析及应用前景,为后续研究立项提供依据(无实验数据)。

\section{引言}
传统音乐系统多依赖 12-TET,等分八度便于转调但牺牲纯律和谐,导致若干音程偏差。受 Petersen 图(3-正则、10 节点)启发,本系统探索几何与哲学混合框架:以图拓扑作为音程约束,结合五行与阴阳生成非线性音阶。五行作为生成—克制循环的象征,可用于音乐疗愈和情感映射。本报告旨在重新构想 Petersen 音乐系统,评估其产生和谐音乐的潜力与可行性。

\section{系统概述}
\subsection{核心框架}
\begin{itemize}
  \item 基于 Petersen 图,定义 15 方位音阶:5 个五行元素在角度上分布为 金 0\textdegree、木 72\textdegree、水 144\textdegree、火 216\textdegree、土 288\textdegree,每元素对应 3 个极性(阴 -1、中 0、阳 +1)。
  \item 阴阳偏移参数 $\Delta\theta$(默认 5\textdegree)用于调整角度:$\theta=\theta_0+\Delta\theta\cdot p$($p\in\{-1,0,1\}$)。
  \item 音高比例采用黄金比例尺度 $\rho=\varphi^{\theta/72}$,$\varphi\approx1.618$。基准频率 $F_0=220\,$Hz,结果折叠到单八度区间以保证听感。
  \item Petersen 图的无向边定义音程语法:旋律按邻接节点游走,形成“受限自由”的作曲机制。
  \item 五行映射到音色与情感(例如:金=钟声/坚定,水=滑音/流动);阴阳调制影响音色亮度(阳倾向明亮谐波,阴倾向暗淡滤波)。
\end{itemize}

\subsection{阴阳偏移参数(默认 5\textdegree)}
推荐默认值 $\Delta\theta=5$\textdegree。理由概述:
\begin{itemize}
  \item 5\textdegree 对应的微分音偏移约 116 cents,能产生温和的微分音效果而非突兀跳变。
  \item 相较于 10\textdegree(约 231 cents),5\textdegree 降低不谐风险,同时保留 72\textdegree(≈833 cents)為五行基間隔。
  \item 参数应可调(例如尝试 18\textdegree 探索“黄金规模”模式),並通过听感测试验证不同设置的影响。
\end{itemize}

\subsection{扩展到 45 音符}
参考 Petersen 图的三环结构:内环/中环/外环分别对应低/中/高音区,采用倍率 $1/\varphi,\,1,\,\varphi$ 将 15 方位复制为 45 音符。系统为动态网络:旋律可在环间转移,形成层级叙事。

\section{创新点与理论基础}
\subsection{规则驱动 vs 数据驱动}
本系统为规则驱动:无需大规模训练样本即可生成音乐;以黄金比例产生非线性音程,并以图论约束减少不协调跳跃。与基于深度学习的生成模型不同,本系统强调可解释的规则与音乐语法。

\subsection{非八度循环潜力}
可选采用“黄金八度”(比值 $\varphi:1$,约 833 cents)替代传统 2:1 八度,产生非周期、持续演化的听感;该思路借鉴 Bohlen--Pierce 等非八度音阶实验。

\subsection{五行与阴阳整合}
将五行生克关系映射到旋律生成规则(例如“木生火”“火克金”)以影响旋律趋向与和声化学;可调节阴阳参数以探索“平衡音乐”与情绪表达。

\section{可行性分析}
\subsection{和谐性潜力}
黄金比例调律在特定配置下可接近纯律音程,结合图约束有望生成有机且悦耳的旋律。45 音符增加和声色彩,但需通过参数优化避免刺耳间隔。

\subsection{实现路径}
建议使用 Python + MIDI(或音频合成库)实现 Petersen 图上的规则游走与黄金比例频率映射。风险为不均匀间隔可能导致不适听感,缓解途径包括路径权重化、和谐优先策略與听觉测试。

\subsection{与 12-TET 的比较}
该系统放弃部分转调自由以换取几何與哲学驱动的和谐表达,适合实验音乐、疗愈音乐與跨学科艺术项目。

\section{潜在应用与扩展}
\begin{itemize}
  \item 音乐创作:生成“五行叙事”作品(包括疗愈类音乐)。
  \item 跨领域:声化数据、艺术装置、新型演奏界面或 MPE 合成器设计。
  \item 扩展方向:与 AI 提示整合(保持规则核心)、探索其他图结构与更多音区。
\end{itemize}

\subsection{专业电子音乐人}
他们如何在工作流中使用自定义调律(.scl/.kbm/.tun)?许多实验/氛围/先锋音乐人会采用微分音或自定义调律以拓展音色与氛围。

下面为实践要点与常见工作流,供制作/演出参考:

\begin{itemize}
  \item 何时采用:在 IDM、Drone、Ambient、现代古典 crossover 等领域常见;主流 EDM/House/Techno 多沿用 12‑TET,但也会用 pitch‑bend/微调实现局部效果。
  \item 目标效果:制造“异域”“古风”或“外星”氛围,或者模仿非西方调式(maqam、raga 等),以及实现更滑动/不稳定但富张力的和声质感。
  \item 常用文件与工具:
    \begin{itemize}
      \item .scl(Scala 音阶文件):定义度数/比值,项数不受 128 限制。
      \item .kbm(键盘映射):将 .scl 映射到 128 个 MIDI 键的规则(起始键、循环方式、cent 偏移)。
      \item .tun(128 项或 SysEx):直接给出每个 MIDI 键的频率或以 SysEx 形式发送到合成器。
      \item 常用工具:Scala(命令行/GUI)、pyfluidsynth / libfluidsynth、DAW 中能加载 Scala/.scl 的合成器或支持 MTS 的插件(注意不同插件/合成器支持细节不同)。
    \end{itemize}
  \item 推荐工作流(可脚本化):
    \begin{enumerate}
      \item 用你的生成器导出 .scl(保留全部度数)或导出已剪枝的频率表。
      \item 为现场或批量渲染生成对应的 .kbm 或 128 项 .tun(若目标合成器期望每键频率,请生成 .tun)。
      \item 在合成器/插件中加载 soundfont/合成器音色后,先加载 tuning(或在播放前通过 SysEx/settuning 应用),再播放/渲染 MIDI。
      \item 若目标 DAW/插件不支持直接加载 .scl,可用脚本将 tuning 嵌入 MIDI(在 track 开头插入 SysEx),或渲染为 audio 采样以供取样器使用。
    \end{enumerate}
  \item 实践提示:
    \begin{itemize}
      \item 先在合成器中小范围测试几度音程,避免一次性加载大量未经试听的调律导致不可控的不适听感。
      \item 对现场演出建议准备回退方案(标准 12‑TET patch),并把 tuning 绑定到特定 MIDI 通道以便快速切换。
      \item 若希望最大兼容性,生成标准 128 项的 .tun(或同时提供 .scl + .kbm)以适配不同 synth/host。
    \end{itemize}
\end{itemize}

\section{结论与建议}
Petersen 音乐系统提供一个原创且可实现的框架,融合图论、五行與黄金比例调律,具有生成和谐新音乐的潜力。建议立项步骤:
\begin{enumerate}
  \item 开发原型;
  \item 进行听觉测试与参数优化;
  \item 开展跨文化比較研究。
\end{enumerate}

\end{document}