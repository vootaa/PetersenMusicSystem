% !TEX TS-program = xelatex
\documentclass{article}

\usepackage{fontspec}
\usepackage{amsmath,amssymb}
\usepackage{xeCJK}
\usepackage[normalem]{ulem} % 提供 \uline 用于中文下划线替代斜体
\usepackage{geometry}      % <-- 修复:确保 geometry 已加载
\geometry{a4paper, margin=25mm}

% 字体(按系统 fc-list 确认)
\setmainfont{Kode Mono}
\setCJKmainfont{Songti SC}[ItalicFont={Songti SC}]
\setCJKmonofont{Songti SC}[ItalicFont={Songti SC}]

% 中文下划线强调命令(替代斜体)
\newcommand{\emphzh}[1]{\uline{#1}}

\title{Petersen 五行阴阳音乐系统 — 底层音律说明}
\author{Abe.Chua (初稿)}
\date{2025-08-25}

\begin{document}
\maketitle

本文件定义音高空间、五行阴阳音阶分配、音色映射、和弦与旋律生成的底层数学与实现要点。设计目标是清晰、参数化并便于程序化实现(MIDI / 合成器)。

\section{符号与常量}
\begin{itemize}
  \item 黄金比例: $\varphi=\dfrac{1+\sqrt{5}}{2}\approx1.6180339887$。
  \item 基频: $F_0$(默认 $220\ \mathrm{Hz}$,可改)。
  \item 五行角度基位: $\theta_e=72^\circ\cdot e,\ e\in\{0,\dots,4\}$(0:金,1:木,2:水,3:火,4:土)。
  \item 阴阳极性: $p\in\{-1,0,+1\}$(阴、中、阳)。
  \item 阴阳角度偏移: $\Delta\theta$(默认 $5^\circ$,可调)。
  \item 折叠基准比(八度选择): $R_{\mathrm{oct}}$,常见取值 $2$ 或 $\varphi$。
\end{itemize}

\section{角度到频率的映射(核心公式)}
先定义该音的角度:
\[
\theta = \theta_e + p\cdot\Delta\theta .
\]
定义原始比例因子(未经折叠):
\[
r_{\mathrm{raw}}(\theta)=\varphi^{\theta/72}.
\]
原始频率:
\[
f_{\mathrm{raw}} = F_{0}\cdot r_{\mathrm{raw}}(\theta) = F_{0}\cdot \varphi^{\theta/72}.
\]
为便于听感,将频率折叠到一个期望区间:
\[
F = f_{\mathrm{raw}}\cdot R_{\mathrm{oct}}^{k},\qquad k\in\mathbb{Z},
\]
其中选择整数 $k$ 使得
\[
F\in [F_{0},\ F_{0}\cdot R_{\mathrm{oct}}).
\]
可取
\[
k = -\left\lfloor \log_{R_{\mathrm{oct}}}\dfrac{f_{\mathrm{raw}}}{F_{0}} \right\rfloor.
\]

\subsection{以 cents 表示}
\[
\mathrm{cents}(F) = 1200\log_2\frac{F}{F_0}.
\]
单位角度对应的 cents 增量为:
\[
\Delta_{\text{1}^\circ} = \frac{1200}{72}\log_2\varphi \approx 11.57\ \text{cents/deg}.
\]

\section{15 方位与 45 音区的定义}
\begin{itemize}
  \item 15 方位:五行 $\times$ 三极性,共 15 个音位,索引表示为 $(e,p)$。
  \item 三音区扩展(45 音符):复制 15 方位到低/中/高区,音区缩放因子 $s\in\{1/\varphi,\,1,\,\varphi\}$(或 $2^{-1},1,2$)。频率:
  \[
  F_{r} = F(e,p)\cdot s_r,\qquad r\in\{L,M,H\}.
  \]
\end{itemize}

\section{五行-阴阳到音色与合成参数的映射}
每个方位映射到一组合成控制参数:
\[
\mathrm{timbre}(e,p)=\{\ \mathrm{instr\_set},\ \mathrm{filter\_cutoff},\ \mathrm{harmonic\_richness},\ \mathrm{env}\ \}.
\]
建议映射(候选乐器/合成器风格):
\begin{itemize}
  \item 金(e=0): 钟/铃类;阳亮,高谐波;阴暗,轻混响。
  \item 木(e=1): 拨弦/木琴;强调基频与击发噪声。
  \item 水(e=2): 垫/木笛;滑音与微分音。
  \item 火(e=3): 铜管/合成铅;强攻与饱和。
  \item 土(e=4): 低音弦/鼓;厚重低频。
\end{itemize}
阴阳参数示例:
\[
\begin{array}{ll}
\text{阳}\ (p=+1):& \ \text{cutoff}+\Delta_c,\ \text{harmonic}+\Delta_h,\ \text{attack}\downarrow\\[2pt]
\text{中}\ (p=0):&\ \text{baseline}\\[2pt]
\text{阴}\ (p=-1):&\ \text{cutoff}-\Delta_c,\ \text{harmonic}-\Delta_h,\ \text{decay}+\Delta_d
\end{array}
\]

\section{和弦设计(规则化生成)}
给定候选频率 $\{F_i\}$,两音间相对于小整数比 $(n:m)$ 的差异:
\[
\delta_{ij}^{(n:m)}=\left|1200\log_2\frac{F_i}{F_j}-1200\log_2\frac{n}{m}\right|.
\]
取 $\delta_{ij}=\min_{(n:m)\in\mathcal{S}}\delta_{ij}^{(n:m)}$,常用 $\mathcal{S}=\{1:1,2:1,3:2,4:3,5:4\}$。
和弦评分:
\[
S(C)=\sum_{i<j}\exp\!\Big(-\frac{\delta_{ij}^2}{2\sigma^2}\Big).
\]

生成策略:选根音 $(e,p,r)$,在 15 方位或相邻音区中搜索候选并计算 $S$,挑选高分 2--3 音为 triad。

\section{旋律生成(底层语法)}
优先小步移动:在同一五行的三极性间或相邻五行 $(e\pm1)\ \mathrm{mod}\ 5$ 移动;偶尔长跳 $(e\pm2)$。阴阳用于微分音与音色修饰。概率模型:
\[
P(\Delta e,\Delta p)\propto \exp\!\big(-\alpha_e|\Delta e|-\alpha_p|\Delta p|\big)\cdot W(e',p'),
\]
其中 $W$ 为音色/和谐权重。

\section{节奏、循环与时间}
基础 BPM = $B$。段落间可按黄金比例调整:$\mathrm{BPM}'=B\cdot\varphi^{\pm q}$。五行到节奏密度示例:金短音,木琶音,水 rubato,火重拍,土长音。micro-timing:
\[
\delta t = \alpha_t \sin\!\big( \tfrac{\pi}{180}\theta \big)\,T_{\mathrm{beat}}.
\]

\section{实现注意事项}
MIDI 实现建议使用 pitch-bend 或 MPE;合成器要求支持可编程滤波、谐波控制、ADSR、portamento。建议批量扫描超参 $\Delta\theta,\ F_0,\ R_{\mathrm{oct}},\ \sigma,\ \alpha_e$ 并导出 MIDI 进行听感测试。

\section{结语}
此文档提供可工程化的底层规范:参数化的角度映射、折叠策略、五行/阴阳到音色与时域行为的映射,以及和弦/旋律生成的评分函数。
下一步可基于此实现最小原型(Python + MIDO 或 SuperCollider),进行主观听觉测试并迭代超参。

\end{document}