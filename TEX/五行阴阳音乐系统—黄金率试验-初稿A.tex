% !TEX TS-program = xelatex
\documentclass{article}

\usepackage{fontspec}
\usepackage{amsmath,amssymb}
\usepackage{xeCJK}
\usepackage[normalem]{ulem}
\usepackage{geometry}
\geometry{a4paper, margin=25mm}

\setmainfont{Kode Mono}
\setCJKmainfont{Songti SC}[ItalicFont={Songti SC}]
\setCJKmonofont{Songti SC}[ItalicFont={Songti SC}]

\newcommand{\emphzh}[1]{\uline{#1}}

\title{五行阴阳音乐系统 — 黄金率试验}
\author{Abe.Chua(初稿A)}
\date{2025-08-25}

\begin{document}
\maketitle

目标(简要):
本试验以可参数化的基点频率 \(F_{\mathrm{base}}\)(默认 20 Hz,可小范围可调)出发,按黄金比例 \(\varphi\) 向上生成一系列音区;在每个音区内按五行—阴阳角度分布决定 15 个音位在该区内的落点。下面给出公式、两种可选映射方案、枚举区间的方法与简单示例。

参数与符号:
\[
\varphi=\frac{1+\sqrt5}{2}\approx1.61803398875,\qquad F_{\mathrm{base}}\ (default=20\ \mathrm{Hz})
\]
五行角度基位与极性:
\[
\theta_e = 72^\circ\cdot e,\ e\in\{0,1,2,3,4\},\qquad p\in\{-1,0,+1\},\qquad \Delta\theta\ (default=5^\circ)
\]
定义单个位置角度:
\[
\theta=\theta_e + p\cdot\Delta\theta.
\]

方案 A — 折叠(folding)方式(与早前文档一致的拓展):
\begin{itemize}
  \item 先计算原始比例因子:
  \[
  r_{\mathrm{raw}}(\theta)=\varphi^{\theta/72}.
  \]
  \item 原始频率(未折叠):
  \[
  f_{\mathrm{raw}} = F_{\mathrm{base}}\cdot r_{\mathrm{raw}}(\theta).
  \]
  \item 将 \(f_{\mathrm{raw}}\) 折叠到第 \(n\) 音区(区间定义见下):
  \[
  f_{e,p,n} = f_{\mathrm{raw}}\cdot\varphi^{k},\qquad k\in\mathbb{Z}
  \]
  选取 \(k\) 使得 \(f_{e,p,n}\in[\,F_{\mathrm{base}}\varphi^{n},\ F_{\mathrm{base}}\varphi^{n+1})\)。
\end{itemize}

方案 B — 归一化(intra-zone placement)方式(推荐,直观且无额外折叠):
将角度映射到单一区间内的位置参数 \(u\in[0,1)\):
\[
u=\frac{\theta \bmod 72^\circ}{72^\circ}.
\]
第 \(n\) 音区定义为
\[
\text{区}_n=[\,F_{\mathrm{base}}\varphi^{n},\ F_{\mathrm{base}}\varphi^{n+1}\,),
\]
并在区内按 \(u\) 放置该音位:
\[
f_{e,p,n} = F_{\mathrm{base}}\cdot \varphi^{\,n+u}.
\]
优点:每 72° 的角度周期对应完整的一个黄金率音区,计算简单且无折叠不确定性。

如何枚举有效的 \(n\)(使频率落在工程带宽 \([F_{\min},F_{\max}]\) 内,示例取 \(F_{\min}=40\ \mathrm{Hz},\ F_{\max}=6000\ \mathrm{Hz}\)):
对于给定 \(u\),满足
\[
F_{\min}\le F_{\mathrm{base}}\varphi^{n+u}<F_{\max}
\]
等价于
\[
\log_{\varphi}\frac{F_{\min}}{F_{\mathrm{base}}}-u \le n < \log_{\varphi}\frac{F_{\max}}{F_{\mathrm{base}}}-u.
\]
因此取
\[
n_{\min}=\left\lceil \log_{\varphi}\frac{F_{\min}}{F_{\mathrm{base}}}-u \right\rceil,\quad
n_{\max}=\left\lfloor \log_{\varphi}\frac{F_{\max}}{F_{\mathrm{base}}}-u \right\rfloor.
\]
对所有 15 个 \((e,p)\) 计算对应 \(u\),再合并不同 \(n\) 的结果即可得到完整候选集合。

简单示例(参数: \(F_{\mathrm{base}}=20\ \mathrm{Hz},\ \Delta\theta=5^\circ,\ F_{\min}=40,\ F_{\max}=6000\)):
按方案 B,枚举 \(n\) 直到区间上界超出 \(F_{\max}\)。(可在脚本中实现并直接导出每个 \(f_{e,p,n}\) 与其所属区间。)

实现建议与比较:
\begin{itemize}
  \item 首选方案 B(归一化)用于生成可解释的、无折叠二义性的区间分布。
  \item 方案 A 可用于兼容已有折叠逻辑或当角度范围超出一个周期时复用。
  \item 导出时同时给出绝对频率、相对 cents(以某一基准如 220 Hz 或局部区下界为基准)、以及所属 \(n\)。
  \item 在脚本中提供可调参数:\(F_{\mathrm{base}},\ \Delta\theta,\ F_{\min},\ F_{\max}\) 与是否使用方案 A/B。
\end{itemize}

下一步(工程化清单):
\begin{enumerate}
  \item 编写生成脚本(Python),实现方案 B 的 \(f_{e,p,n}\) 枚举并输出 CSV/Scala/.tun。
  \item 生成每个候选区段的频率范围表并可视化(热图/频谱带图)。
  \item 在小样本上进行听测与参数微调(调整 \(F_{\mathrm{base}}\) 与 \(\Delta\theta\))。
\end{enumerate}

\end{document}