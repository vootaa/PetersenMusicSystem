% !TEX TS-program = xelatex
\documentclass{article}

\usepackage{fontspec}
\usepackage{amsmath,amssymb}
\usepackage{xeCJK}
\usepackage[normalem]{ulem}
\usepackage{geometry}
\geometry{a4paper, margin=25mm}

\setmainfont{Kode Mono}
\setCJKmainfont{Songti SC}[ItalicFont={Songti SC}]
\setCJKmonofont{Songti SC}[ItalicFont={Songti SC}]

\newcommand{\emphzh}[1]{\uline{#1}}

\title{Petersen 五行阴阳音乐系统 — 底层音律说明}
\author{Abe.Chua (初稿B)}
\date{2025-08-25}

\begin{document}
\maketitle

本文件为底层音律说明的升级版,补充了可复现的参数表、实现/导出实践、测试协议与最小化生成脚本。目标:便于工程实现(Python/SC/SuperCollider)、听觉验证与 DAW/合成器集成。

\section{符号与常量(概览)}
简要重述主要符号:
\begin{itemize}
  \item 黄金比例: $\varphi=\dfrac{1+\sqrt{5}}{2}\approx1.6180339887$。
  \item 基频: $F_0$(默认 $220\ \mathrm{Hz}$)。
  \item 五行角度基位: $\theta_e=72^\circ\cdot e,\ e\in\{0,\dots,4\}$(0:金,1:木,2:水,3:火,4:土)。
  \item 阴阳极性: $p\in\{-1,0,+1\}$;阴阳偏移: $\Delta\theta$(默认 $5^\circ$)。
  \item 折叠基准比(八度选择): $R_{\mathrm{oct}}\in\{2,\varphi\}$。
\end{itemize}

\section{标准参数表(默认值与可搜索范围)}
为保证复现性与方便扫参,建议在文档或配置文件明确列出默认值与实验范围:
\begin{itemize}
  \item F\_0 = 220 Hz (可试 160–440 Hz)
  \item $\Delta\theta$ = 5° (可试 0°–18°,或特殊值如 10°, 18°)
  \item $R_{\mathrm{oct}}$ = 2(可选 $\varphi$)
  \item $\sigma$(和弦相容性宽度)= 50 cents(可试 10–200 cents)
  \item $\alpha_e$(音行跳跃惩罚)= 1.0(可试 0.1–5.0)
  \item BPM 基准 $B$(实验值按需要)
  \item MPE pitch‑bend range 建议初期 ±200 cents(根据最大偏移调整)
\end{itemize}

\section{角度到频率—实现细节(复述与边界)}
核心映射公式与折叠策略如前文所述。实现时注意:
\begin{itemize}
  \item 使用高精度浮点(double)计算 phi 的幂与对数,避免累积误差。
  \item 折叠到区间 $[F_0, F_0\cdot R_{\mathrm{oct}})$,通过
  \[
    k = -\left\lfloor \log_{R_{\mathrm{oct}}}\dfrac{f_{\mathrm{raw}}}{F_{0}} \right\rfloor
  \]
  保证单一标准区间。
  \item 输出 cents 时采用基准 F0 的相对 cents: $\mathrm{cents}=1200\log_2(F/F_0)$。
\end{itemize}

\section{示例:15 方位(F\_0=220Hz, \(\Delta\theta=5^\circ\), \(R_{\mathrm{oct}}=2\))}
下表为示例计算(折叠后频率均落入 [220,440))。可将此表纳入 README 或导出文件以便听评。

\begin{itemize}
  \item 金 (e=0)
    \begin{itemize}
      \item p = -1: F ≈ 425.07 Hz, cents ≈ 1140
      \item p =  0: F = 220.00 Hz, cents = 0
      \item p = +1: F ≈ 227.47 Hz, cents ≈ 58
    \end{itemize}
  \item 木 (e=1)
    \begin{itemize}
      \item p = -1: F ≈ 344.28 Hz, cents ≈ 775
      \item p =  0: F ≈ 355.99 Hz, cents ≈ 833
      \item p = +1: F ≈ 368.20 Hz, cents ≈ 891
    \end{itemize}
  \item 水 (e=2)
    \begin{itemize}
      \item p = -1: F ≈ 278.70 Hz, cents ≈ 410
      \item p =  0: F ≈ 287.98 Hz, cents ≈ 468
      \item p = +1: F ≈ 297.66 Hz, cents ≈ 523
    \end{itemize}
  \item 火 (e=3)
    \begin{itemize}
      \item p = -1: F ≈ 225.27 Hz, cents ≈ 42
      \item p =  0: F ≈ 232.98 Hz, cents ≈ 99
      \item p = +1: F ≈ 242.75 Hz, cents ≈ 170
    \end{itemize}
  \item 土 (e=4)
    \begin{itemize}
      \item p = -1: F ≈ 364.25 Hz, cents ≈ 871
      \item p =  0: F ≈ 376.98 Hz, cents ≈ 933
      \item p = +1: F ≈ 388.78 Hz, cents ≈ 986
    \end{itemize}
\end{itemize}

45 音区通过对上表分别乘以 $s\in\{1/\varphi,1,\varphi\}$(或 $2^{-1},1,2$)生成;建议导出 CSV/Scala/.tun 三种格式。

\section{导出格式与 DAW/合成器 集成实践}
建议同时支持三类输出:
\begin{enumerate}
  \item Scala (.scl):用于静态音阶比较、格式通用。
  \item MIDI Tuning Standard (.tun) 或 SysEx:用于支持自定义微分音调律的合成器。
  \item MPE + per‑voice pitch‑bend fallback:当合成器支持 MPE 时,每声部使用独立 pitch‑bend;否则用 base MIDI note + pitch‑bend。建议策略:
    \begin{itemize}
      \item 计算目标频率与最邻近 MIDI note(等分十二平均律)差值(cents)。
      \item 若绝对差值 $\le$ 100 cents,分配该 MIDI note 並通过 pitch‑bend 调整;否则选择更合适 octave / base note。
      \item 默认 pitch‑bend range = ±200 cents;如果系统需要更大范围,动态调整或选择不同基准 note 组合以压缩到范围内。
    \end{itemize}
\end{enumerate}

\section{和弦评分、搜索与约束(实现细化)}
建议实现要点:
\begin{itemize}
  \item 在计算 $\delta_{ij}$ 时同时记录对应的小整数比(n:m),用于后续分析与可视化。
  \item 对于多音和弦,引入基频对齐惩罚(若两个音的基频接近简单倍頻关系則加分)。
  \item 搜索时采取分层策略:先在同一五行/相邻五行中枚舉候選,再按和弦分數 S(C) 排序,最後對時間窗口內的平滑性進行二次篩選(避免瞬時 cluster)。
\end{itemize}

\section{实时合成与抗失真注意}
实现时需注意:
\begin{itemize}
  \item 频率变化平滑(滑音/portamento):对频率包络应用分段线性或低通滤波,避免瞬时大跳導致 aliasing。
  \item 采样率与插值:使用至少 48 kHz,带线性相位低通滤波器的 resampler 用于 pitch‑bend 引起的频率变化。
  \item 预计算 LUT:在实时环境中预先生成 15/45 音频频率表与 pitch‑bend 值,减少计算负担。
\end{itemize}

\section{可视化、调试与听觉评估流程}
最小可行试验(MVP)流程:
\begin{enumerate}
  \item 使用默认参数生成 15/45 的 Scala 与 MIDI 文件。
  \item 在受控条件下进行主观 A/B 听测(建议 20–30 位受试者),并记录偏好。
  \item 同步记录客观指标:平均偏差(cents)、最大偏差、与常见小整数比的分布。
  \item 用热力图显示 Petersen 图上节点被访问频度、和弦评分分布与路径转移矩阵。
\end{enumerate}

\section{音色补丁建议(工程化)}
为便于快速试验,提供五行到合成参数的映射示例(仅参数示范,实际数值按合成器调整):
\begin{itemize}
  \item 金:bell FM / high-q resonant filter, bright harmonic boost。
  \item 木:plucked sample 或 physical model, transient + body noise。
  \item 水:pad + portamento, low-pass with subtle chorus。
  \item 火:brass/lead with saturation, shorter release, high harmonic content。
  \item 土:low oscillator, heavy LP filter, long decay。
\end{itemize}
阴阳可映射为:brightness, harmonic content, attack/decay 微调。

\section{测试与参数搜索建议(自动化)}
实现时建议:
\begin{itemize}
  \item 网格或贝叶斯优化搜索 $\Delta\theta,\ \sigma,\ \alpha_e,\ R_{\mathrm{oct}}$。
  \item 每组参数自动生成一组 MIDI/Audio,记录客观指标并做批量 A/B(或对照 12‑TET)听测。
  \item 保存所有结果与元数据(参数 JSON + timestamp + render 文件名),便于后续分析与复现。
\end{itemize}

\section{元数据、引用与许可证}
请在项目中添加术语表与参考文献(Bohlen–Pierce, Sevish, microtonal tuning 文献),并声明代码与数据许可证(建议 MIT/CC-BY 兼容组合)。

\section{附录 A:生成与导出最小 Python 脚本(工程化示例)}
下列脚本为最小可运行示例:生成 15 音位表,导出 CSV 与 Scala (.scl)。请将脚本保存在 \texttt{tools/generate\_tuning.py} 并在 macOS 终端运行 \texttt{python3}。

\begin{verbatim}
# filepath: tools/generate_tuning.py
# 运行: python3 tools/generate_tuning.py
import math, csv

phi = (1+5**0.5)/2
F0 = 220.0
dtheta = 5.0
R_oct = 2.0

elements = ['金','木','水','火','土']
rows = []

def raw_freq(theta):
    return F0 * (phi ** (theta/72.0))

def fold_freq(f):
    k = -math.floor(math.log(f / F0, R_oct))
    return f * (R_oct ** k)

def cents(f):
    return 1200*math.log2(f / F0)

for e_idx, name in enumerate(elements):
    theta_e = 72.0 * e_idx
    for p in (-1,0,1):
        theta = theta_e + p*dtheta
        f_raw = raw_freq(theta)
        F = fold_freq(f_raw)
        rows.append({
            'element': name,
            'e': e_idx,
            'p': p,
            'theta': theta,
            'freq': round(F,6),
            'cents': round(cents(F),3)
        })

# CSV
with open('tools/15_positions.csv','w', newline='') as f:
    w = csv.DictWriter(f, fieldnames=['element','e','p','theta','freq','cents'])
    w.writeheader()
    w.writerows(rows)

# Scala .scl (single octave relative to F0)
with open('tools/petersen_15.scl','w') as f:
    f.write("! petersen_15.scl\n")
    f.write("Petersen 五行阴阳 15-tone scale (F0=220Hz, dtheta=5deg)\n")
    f.write("15\n")
    for r in rows:
        cents_val = r['cents']
        f.write(f"{cents_val}\n")
print("Generated tools/15_positions.csv and tools/petersen_15.scl")
\end{verbatim}

\section{结语}
本升级版补充了工程实现所需的参数表、导出实践、最小脚本与听觉测试流程。下一步建议:
\begin{enumerate}
  \item 在 tools 中运行脚本並验证 15/45 导出;
  \item 在 DAW(支持 MPE 或 .tun)中做对照听测;
  \item 启动参数扫参并记录结果。
\end{enumerate}

\end{document}