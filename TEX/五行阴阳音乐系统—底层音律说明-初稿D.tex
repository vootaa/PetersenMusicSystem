% !TEX TS-program = xelatex
\documentclass{article}

\usepackage{fontspec}
\usepackage{amsmath,amssymb}
\usepackage{xeCJK}
\usepackage[normalem]{ulem}
\usepackage{geometry}
\geometry{a4paper, margin=25mm}

\setmainfont{Kode Mono}
\setCJKmainfont{Songti SC}[ItalicFont={Songti SC}]
\setCJKmonofont{Songti SC}[ItalicFont={Songti SC}]

\newcommand{\emphzh}[1]{\uline{#1}}

\title{Petersen 五行阴阳音乐系统 — 底层音律说明}
\author{Abe.Chua (初稿D)}
\date{2025-08-25}

\begin{document}
\maketitle

本文件在初稿C基础上,增加了钢琴频率和可选的音区范围,基本思路是不局限于原先设定的三个音区。
主要依据:人耳可接受频率范围:约 20Hz – 20kHz,但音乐习惯主要在 40Hz – 6kHz。

\section{概览与设计目标}
本系统以黄金比例为音高生成基准、以 Petersen 图与五行/阴阳为结构化语法,目标是:
\begin{itemize}
  \item 参数化、可复现的音律生成(15 方位,扩展至多个音区)。
  \item 便于导出至 Scala/.scl、MIDI Tuning (.tun/SysEx) 与 MPE 友好输出。
  \item 支持和弦评分、搜索、实时合成与听觉评估流程。
\end{itemize}

\section{符号与常量(总览)}
主要符号与默认常量:
\begin{itemize}
  \item 黄金比例: $\varphi=\dfrac{1+\sqrt{5}}{2}\approx1.6180339887$。
  \item 基频: $F_0$(默认 $220\ \mathrm{Hz}$)。
  \item 五行角度基位: $\theta_e=72^\circ\cdot e,\ e\in\{0,\dots,4\}$(0:金,1:木,2:水,3:火,4:土)。
  \item 阴阳极性: $p\in\{-1,0,+1\}$;阴阳角度偏移: $\Delta\theta$(默认 $5^\circ$)。
  \item 折叠基准比(八度选择): $R_{\mathrm{oct}}\in\{2,\varphi\}$。
  \item 其他超参:和弦相容宽度 $\sigma$、音行跳跃惩罚 $\alpha_e$、BPM 基准 $B$、MPE pitch-bend range 等。
\end{itemize}

\section{标准参数表(默认值与可搜索范围)}
建议在实现配置中明确列出默认值与搜索范围:
\begin{itemize}
  \item F\_0 = 220 Hz (可试 160–440 Hz)
  \item $\Delta\theta$ = 5° (可试 0°–18°;特殊:1.125°,2.25°,4.5°,9°,18°=72°/4 )
  \item $R_{\mathrm{oct}}$ = 2(可选 $\varphi$)
  \item $\sigma$ = 50 cents(可试 10–200 cents)
  \item $\alpha_e$ = 1.0(可试 0.1–5.0)
  \item MPE pitch-bend range 建议 ±200 cents(按需要调整)
\end{itemize}

\section{钢琴频率与标准音区}
标准 88 键钢琴频率范围:
\begin{itemize}
  \item 最低音:A0 = 27.500 Hz
  \item 最高音:C8 ≈ 4186.009 Hz
  \item 参考:A4 = 440 Hz(等温平均律)
\end{itemize}

为与本体系的“标准音区”保持一致,文稿采用标准音区(基准区间):
\[
\text{标准音区} = [440,\ 880)\ \text{Hz}.
\]

\subsection{按 \(\varphi=2\)(即因子为 \(2^{n}\))扩展 — 候选区间}
以基准区间 $[440,880)$ 为中心,按 $s=2^{n}$ 向两侧扩展):

\begin{itemize}
  \item $s=1/16$:$[27.500,\ 55.000]$
  \item $s=1/8$ :$[55.000,\ 110.000]$
  \item $s=1/4$ :$[110.000,\ 220.000]$
  \item $s=1/2$ :$[220.000,\ 440.000]$
  \item $s=1$   :$[440.000,\ 880.000]$
  \item $s=2$   :$[880.000,\ 1760.000]$
  \item $s=4$   :$[1760.000,\ 3520.000]$
  \item $s=8$   :$[3520.000,\ 7040.000]$
\end{itemize}

结论 ($\varphi=2$ 情形):可得到 8 个候选区间。

\subsection{按黄金比例 \(\varphi\approx1.61803398875\) 扩展 — 候选区间}
以基准区间 $[440,880)$ 为中心,按 $s=\varphi^{n}$(整数 $n$)扩展。取 $n\in\{-6,\dots,5\}$:

\begin{itemize}
  \item $n=-5$ ($s\approx0.090170$):$[39.675,\ 79.350]$
  \item $n=-4$ ($s\approx0.145898$):$[64.195,\ 128.390]$
  \item $n=-3$ ($s\approx0.236068$):$[103.867,\ 207.735]$
  \item $n=-2$ ($s\approx0.381966$):$[167.667,\ 335.334]$
  \item $n=-1$ ($s\approx0.618034$):$[271.934,\ 543.869]$
  \item $n=0$  ($s=1$)          :$[440.000,\ 880.000]$
  \item $n=1$  ($s\approx1.618034$):$[711.935,\ 1423.869]$
  \item $n=2$  ($s\approx2.618034$):$[1151.935,\ 2303.869]$
  \item $n=3$  ($s\approx4.236068$):$[1863.869,\ 3727.736]$
  \item $n=4$  ($s\approx6.854102$):$[3015.805,\ 6031.611]$
\end{itemize}

\subsection{按黄金比例 \(\varphi\approx1.61803398875\) 扩展 — 候选区间}
采用黄金比例生成音区时,不以440/880为基准,而以20Hz为最低基点,向上按因子 \(\varphi\) 逐步扩展至区间接近6000Hz附近。
\\每个音区定义为:
\[
\text{区}_n = \big[\,20\varphi^{n},\ 20\varphi^{n+1}\,\big),\qquad n=0,1,2,\dots
\]

计算结果(保留三位小数):
\begin{itemize}
  \item $n=0$:$[20.000,\ 32.361)\ \mathrm{Hz}$
  \item $n=1$:$[32.361,\ 52.361)\ \mathrm{Hz}$
  \item $n=2$:$[52.361,\ 84.721)\ \mathrm{Hz}$
  \item $n=3$:$[84.721,\ 137.082)\ \mathrm{Hz}$
  \item $n=4$:$[137.082,\ 221.803)\ \mathrm{Hz}$
  \item $n=5$:$[221.803,\ 358.885)\ \mathrm{Hz}$
  \item $n=6$:$[358.885,\ 580.689)\ \mathrm{Hz}$
  \item $n=7$:$[580.689,\ 939.574)\ \mathrm{Hz}$
  \item $n=8$:$[939.574,\ 1520.263)\ \mathrm{Hz}$
  \item $n=9$:$[1520.263,\ 2459.837)\ \mathrm{Hz}$
  \item $n=10$:$[2459.837,\ 3980.100)\ \mathrm{Hz}$
  \item $n=11$:$[3980.100,\ 6439.936)\ \mathrm{Hz}$
\end{itemize}

结论:以 20 Hz 为基点并按黄金比例向上扩展可得到 12 个候选音区集合,或按需要合并/细分以匹配15方位映射与实际听觉/合成器约束。

\section{角度—频率映射(核心公式)}
定义音位角度:
\[
\theta = \theta_e + p\cdot\Delta\theta .
\]
原始比例因子:
\[
r_{\mathrm{raw}}(\theta)=\varphi^{\theta/72}.
\]
原始频率:
\[
f_{\mathrm{raw}} = F_{0}\cdot \varphi^{\theta/72}.
\]
折叠到期望区间(单一标准区间):
\[
F = f_{\mathrm{raw}}\cdot R_{\mathrm{oct}}^{k},\qquad k\in\mathbb{Z},
\]
其中可取
\[
k = -\left\lfloor \log_{R_{\mathrm{oct}}}\dfrac{f_{\mathrm{raw}}}{F_{0}} \right\rfloor
\]
以保证
\[
F\in [F_{0},\ F_{0}\cdot R_{\mathrm{oct}}).
\]
以 cents 表示:
\[
\mathrm{cents}(F) = 1200\log_2\frac{F}{F_0}.
\]
每度的 cents 增量:
\[
\Delta_{1^\circ} = \frac{1200}{72}\log_2\varphi \approx 11.57\ \text{cents/deg}.
\]

实现注意:
\begin{itemize}
  \item 使用高精度 double 计算 phi 幂与对数,避免累积误差。
  \item 若选择 $R_{\mathrm{oct}}=\varphi$(“黄金八度”),需在导出/播放链路中处理非常规八度映射。
\end{itemize}

\section{五行阴阳5*3方位与多音区}
定义:
\begin{itemize}
  \item 15 方位:五行(5)× 三极性(3)= 15 个音位,索引 (e,p)。
  \item 多个音区:复制 15 方位到多个高区,音区缩放因子 $s\in\{1/\varphi,\,1,\,\varphi\}$(或 $2^{-1},1,2$)生成更丰富的音高集合。
  \item 基本音确定为金阴,还是金中,取决于未来归一化处理后,确立对应音区后设定。
\end{itemize}

\section{导出格式与 DAW/合成器 集成实践}
建议同时支持三类输出:
\begin{enumerate}
  \item Scala (.scl):用于静态音阶比较、格式通用。
  \item MIDI Tuning Standard (.tun) 或 SysEx:用于支持自定义微分音调律的合成器。
  \item MPE + per-voice pitch-bend fallback:当合成器支持 MPE 时,每声部使用独立 pitch-bend;否则用 base MIDI note + pitch-bend。策略建议:
    \begin{itemize}
      \item 计算目标频率与最邻近 MIDI note(12-TET)差值(cents)。
      \item 若绝对差值 $\le$ 100 cents,分配该 MIDI note 並通过 pitch-bend 调整;否则选择不同基准 note/八度组合以压缩到 pitch-bend 范围。
      \item 默认 pitch-bend range = ±200 cents;如需更大范围,动态调整或采用 MPE 每声部方案。
    \end{itemize}
\end{enumerate}

\section{和弦评分、搜索与约束(实现细化)}
两音间相对于小整数比 $(n:m)$ 的差异:
\[
\delta_{ij}^{(n:m)}=\left|1200\log_2\frac{F_i}{F_j}-1200\log_2\frac{n}{m}\right|.
\]
取
\[
\delta_{ij}=\min_{(n:m)\in\mathcal{S}}\delta_{ij}^{(n:m)},
\]
常用集合 $\mathcal{S}=\{1:1,2:1,3:2,4:3,5:4\}$(可扩展)。
和弦评分:
\[
S(C)=\sum_{i<j}\exp\!\Big(-\frac{\delta_{ij}^2}{2\sigma^2}\Big).
\]

实现建议:
\begin{itemize}
  \item 在计算 $\delta_{ij}$ 时同时记录对应的小整数比(n:m)用于分析与可视化。
  \item 对多音和弦引入基频对齐惩罚(若两个音接近简单倍频关系则加分)。
  \item 搜索采取分层策略:先在同一五行/相邻五行中枚举候选,再按和弦分数 S(C) 排序,最后做时间窗口内平滑性二次筛选以避免瞬时 cluster。
\end{itemize}

\section{旋律生成与 Petersen 图语法}
系统将 Petersen 图或其它图结构作为音程/转移约束:
\begin{itemize}
  \item 将 15 方位映射到图节点,使用图的边定义允许的音高转移(例如邻接优先、重复惩罚、五行生克偏好)。
  \item 概率模型示例:
  \[
  P(\Delta e,\Delta p)\propto \exp\!\big(-\alpha_e|\Delta e|-\alpha_p|\Delta p|\big)\cdot W(e',p'),
  \]
  其中 $W$ 为音色/和谐权重,$\alpha_e,\alpha_p$ 为跳跃惩罚参数。
  \item 优先小步移动(同一五行三极性或相邻五行),偶尔长跳以增加张力。
\end{itemize}

\section{实时合成、抗失真与实现注意}
实时实现注意点:
\begin{itemize}
  \item 频率变化平滑(portamento/滑音):对频率包络应用分段线性或低通滤波,避免瞬时大跳引起 aliasing。
  \item 采样率与插值:使用至少 48 kHz;对 pitch-bend 引起的频率变化使用带线性相位低通滤波器的 resampler。
  \item 预计算 LUT:在实时环境中预生成 15/45 音频频率表与 pitch-bend 值,降低运行时负载。
  \item 合成器 patch 建议(映射五行到音色):
    \begin{itemize}
      \item 金:bell FM / high-q resonant filter, bright harmonic boost。
      \item 木:plucked sample / physical model, transient + body noise。
      \item 水:pad + portamento, low-bend with subtle chorus。
      \item 火:brass/lead with saturation, shorter release。
      \item 土:low oscillator, heavy LP filter, long decay。
    \end{itemize}
\end{itemize}

\section{可视化、调试与听觉评估流程}
推荐最小可行试验(MVP)流程:
\begin{enumerate}
  \item 使用默认参数生成 15/45 的 Scala 与 MIDI 文件(见附录最小脚本)。
  \item 在受控条件下进行主观 A/B 听测(建议 20–30 位受试者),并记录偏好。
  \item 同步记录客观指标:平均偏差(cents)、最大偏差、与常见小整数比的分布。
  \item 用热力图显示 Petersen 图上节点被访问频度、和弦评分分布与路径转移矩阵。
\end{enumerate}

\section{测试、参数搜索与自动化建议}
建议的自动化策略:
\begin{itemize}
  \item 使用网格或贝叶斯优化搜索 $\Delta\theta,\ \sigma,\ \alpha_e,\ R_{\mathrm{oct}}$。
  \item 每组参数自动生成一组 MIDI/Audio,记录客观指标并进行批量 A/B(或对照 12-TET)听测。
  \item 保存所有结果与元数据(参数 JSON + timestamp + render 文件名),便于复现与分析。
\end{itemize}

\section{元数据、引用与许可证}
请在项目中添加术语表与参考文献(例如 Bohlen–Pierce, Sevish, microtonal tuning 文献),并声明代码与数据许可证(建议 MIT + CC-BY 组合以兼容科研与艺术共享)。

\section{附录 A:实现注意的细节与实践清单}
实现时的快速清单:
\begin{itemize}
  \item 使用 double 精度计算 phi 的幂与对数,避免累积误差。
  \item 折叠到区间 $[F_0, F_0\cdot R_{\mathrm{oct}})$ 时使用上文公式计算 k。
  \item 当导出 cents 时采用基准 F0 的相对 cents: $\mathrm{cents}=1200\log_2(F/F_0)$。
  \item 提前生成 LUT 与 Scala/.tun 的 mapping,测试合成器对非常规八度(若使用 $\varphi$ 八度)的支持。
\end{itemize}

\section{附录 B:生成与导出最小 Python 脚本(工程化示例)}
下列脚本为最小可运行示例:生成 15 音位表,导出 CSV 与 Scala (.scl)。将脚本保存在 \texttt{tools/generate\_tuning.py} 并在 macOS 终端运行 \texttt{python3}。

\begin{verbatim}
# filepath: tools/generate_tuning.py
# 运行: python3 tools/generate_tuning.py
import math, csv

phi = (1+5**0.5)/2
F0 = 220.0
dtheta = 5.0
R_oct = 2.0

elements = ['金','木','水','火','土']
rows = []

def raw_freq(theta):
    return F0 * (phi ** (theta/72.0))

def fold_freq(f):
    k = -math.floor(math.log(f / F0, R_oct))
    return f * (R_oct ** k)

def cents(f):
    return 1200*math.log2(f / F0)

for e_idx, name in enumerate(elements):
    theta_e = 72.0 * e_idx
    for p in (-1,0,1):
        theta = theta_e + p*dtheta
        f_raw = raw_freq(theta)
        F = fold_freq(f_raw)
        rows.append({
            'element': name,
            'e': e_idx,
            'p': p,
            'theta': theta,
            'freq': round(F,6),
            'cents': round(cents(F),3)
        })

# CSV
with open('tools/15_positions.csv','w', newline='') as f:
    w = csv.DictWriter(f, fieldnames=['element','e','p','theta','freq','cents'])
    w.writeheader()
    w.writerows(rows)

# Scala .scl (single octave relative to F0)
with open('tools/petersen_15.scl','w') as f:
    f.write("! petersen_15.scl\n")
    f.write("Petersen 五行阴阳 15-tone scale (F0=220Hz, dtheta=5deg)\n")
    f.write("15\n")
    for r in rows:
        cents_val = r['cents']
        f.write(f"{cents_val}\n")
print("Generated tools/15_positions.csv and tools/petersen_15.scl")
\end{verbatim}

\section{结语与下一步建议}
本合并稿(初稿D)整合数学定义、参数表、实现细化、导出策略与最小脚本。下一步建议:
\begin{enumerate}
  \item 将附录脚本放入 tools 并运行以验证 15/45 导出;
  \item 在支持 MPE 或 .tun 的 DAW/合成器中进行对照听测;
  \item 启动参数扫参并记录结果(自动化保存 metadata)。
\end{enumerate}

\end{document}