% !TEX TS-program = xelatex
\documentclass{article}

\usepackage{fontspec}
\usepackage{amsmath,amssymb}
\usepackage{xeCJK}
\usepackage[normalem]{ulem}
\usepackage{geometry}
\geometry{a4paper, margin=25mm}

% 段落间距与缩进:使段落更清晰
\setlength{\parskip}{6pt}
\setlength{\parindent}{0pt}

\setmainfont{Kode Mono}
\setCJKmainfont{Songti SC}[ItalicFont={Songti SC}]
\setCJKmonofont{Songti SC}[ItalicFont={Songti SC}]

\newcommand{\emphzh}[1]{\uline{#1}}

\title{五行阴阳音乐系统 — 黄金率试验}
\author{Abe.Chua(初稿B)}
\date{2025-08-25}

\begin{document}
\maketitle

\section*{目标(简要)}
本试验以可参数化的基点频率 \(F_{\mathrm{base}}\)(默认 20\,Hz,调整范围视为转调,小于1个 \(\varphi\))为出发点,按黄金比例 \(\varphi\) 向上生成一系列音区;在每个音区内按五行—阴阳角度分布决定 15 个音位在该区内的落点。
\\下面给出公式、两种可选映射方案、枚举区间的方法与简单示例。

\section*{参数与符号}
主要符号定义如下:
\[
\varphi=\frac{1+\sqrt5}{2}\approx1.61803398875,\qquad F_{\mathrm{base}}\ (\text{default}=20\ \mathrm{Hz}).
\]
五行角度基位与极性定义为:
\[
\theta_e = 72^\circ\cdot e,\quad e\in\{0,1,2,3,4\},\qquad p\in\{-1,0,+1\},\qquad \Delta\theta\ (\text{default}=72^\circ/15=4.8^\circ).
\]

令 p=-1 对应基准角(即 \(\theta_e\) 本身),向上按步进 \(\Delta\theta\) 配置三极性位置:
\[
\theta=\theta_e + (p+1)\cdot\Delta\theta,\qquad p\in\{-1,0,+1\}.
\]

\section*{方案 A — 折叠(folding)方式}
方案 A 是对早前文档折叠逻辑的扩展,适用于需要把原始频率折叠回目标音区的情况。

先计算原始比例因子:
\[
r_{\mathrm{raw}}(\theta)=\varphi^{\theta/72}.
\]

原始频率(未折叠)为
\[
f_{\mathrm{raw}} = F_{\mathrm{base}}\cdot r_{\mathrm{raw}}(\theta).
\]

将 \(f_{\mathrm{raw}}\) 折叠到第 \(n\) 音区(区间定义见下):
\[
f_{e,p,n} = f_{\mathrm{raw}}\cdot\varphi^{k},\qquad k\in\mathbb{Z},
\]
其中选取 \(k\) 使得
\[
f_{e,p,n}\in[\,F_{\mathrm{base}}\varphi^{n},\ F_{\mathrm{base}}\varphi^{n+1}).
\]

方案 A 适合于兼容已有折叠逻辑或当角度范围超出一个周期时复用,但会引入折叠带来的二义性。

\section*{方案 B — 归一化(intra-zone placement)方式(推荐)}
方案 B 将角度映射到单一区间内的位置参数 \(u\in[0,1)\),从而在每个音区内按相对位置放置音位——此法直观且无额外折叠。

角度到区内位置的映射为:
\[
u=\frac{\theta \bmod 72^\circ}{72^\circ}.
\]

第 \(n\) 音区定义为
\[
\text{区}_n=[\,F_{\mathrm{base}}\varphi^{n},\ F_{\mathrm{base}}\varphi^{n+1}\,).
\]

在区内按 \(u\) 放置该音位:
\[
f_{e,p,n} = F_{\mathrm{base}}\cdot \varphi^{\,n+u}.
\]

方案 B 的优点在于每 72° 的角度周期对应完整的一个黄金率音区,计算简单且无折叠不确定性。因此推荐优先使用方案 B 生成可解释的区间分布。

\section*{如何枚举有效的 \(n\)}
为了使频率落在工程带宽 \([F_{\min},F_{\max}]\) 内(示例取 \(F_{\min}=30\ \mathrm{Hz},\ F_{\max}=6000\ \mathrm{Hz}\)),对于给定 \(u\) 需要满足:
\[
F_{\min}\le F_{\mathrm{base}}\varphi^{n+u}<F_{\max}.
\]
等价地,
\[
\log_{\varphi}\frac{F_{\min}}{F_{\mathrm{base}}}-u \le n < \log_{\varphi}\frac{F_{\max}}{F_{\mathrm{base}}}-u.
\]
因此可以取
\[
n_{\min}=\left\lceil \log_{\varphi}\frac{F_{\min}}{F_{\mathrm{base}}}-u \right\rceil,\quad
n_{\max}=\left\lfloor \log_{\varphi}\frac{F_{\max}}{F_{\mathrm{base}}}-u \right\rfloor.
\]
对所有 15 个 \((e,p)\) 计算对应的 \(u\),再合并不同 \(n\) 的结果即可得到完整候选集合。

\section*{简单示例}
参数示例: \(F_{\mathrm{base}}=20\ \mathrm{Hz},\ \Delta\theta=5^\circ,\ F_{\min}=30\ \mathrm{Hz},\ F_{\max}=6000\ \mathrm{Hz}\)。

按方案 B 枚举 \(n\):从最小的 \(n_{\min}\) 开始,对每个 \(u\) 逐步增加 \(n\) 直到区间上界超出 \(F_{\max}\)。该过程在脚本中易于实现并可直接导出每个 \(f_{e,p,n}\) 与其所属区间。

\section*{实现建议与比较}
首选方案 B(归一化)用于生成可解释的、无折叠二义性的区间分布。方案 A 可用于兼容已有折叠逻辑或当角度范围超出一个周期时复用。

导出时建议同时给出:绝对频率、相对 cents(以某一基准如 220\,Hz 或局部区下界为基准)、以及所属 \(n\)。在脚本中提供可调参数:\(F_{\mathrm{base}},\ \Delta\theta,\ F_{\min},\ F_{\max}\) 与是否使用方案 A/B。

\section*{下一步(工程化清单)}
\begin{enumerate}
  \item 编写生成脚本(Python),实现方案 B 的 \(f_{e,p,n}\) 枚举并输出 CSV/Scala/.tun。
  \item 生成每个候选区段的频率范围表并可视化(热图/频谱带图)。
  \item 在小样本上进行听测与参数微调(调整 \(F_{\mathrm{base}}\) 与 \(\Delta\theta\))。
\end{enumerate}

\end{document}